\documentclass[a4paper,11pt]{article}
\usepackage{graphicx}
\usepackage[utf8]{inputenc}
\usepackage{hyperref}
\usepackage{placeins}
\usepackage[newfloat]{minted}
\usepackage{caption}

\newenvironment{code}{\captionsetup{type=listing}}{}
\SetupFloatingEnvironment{listing}{name=Code Overview}


\hypersetup{
    colorlinks=true,
    linkcolor=blue,
    filecolor=black,      
    urlcolor=blue,
    citecolor=black,
}
\begin{document}

\title{
    \textbf{A Heap or Priority Queue Report.}
}
\author{Adrian Jonsson Sjödin}
\date{Fall 2022}

\maketitle

\section{Introduction}
\label{intro}


\section{Task}
\label{task}
The task in this assignment is to create a priority queue using different implementations and determine
the pros and cons of each. The different priority queues to be implemented are listed bellow.
\begin{itemize}
    \item List implementation of priority queue where smaller numbers have higher priority. Make one implementation
    where the cost of adding elements are $\mathcal{O}(1)$ and remove is $\mathcal{O}(n)$, and another where the costs are reversed.
    Benchmark the two implementations and discuss situations where one would be preferred over the other.
    
    \item Priority queue implementing a heap as a linked binary tree where each node holds the element, branches and 
    the total number of elements in the subtree. This to make the tree balanced. You should be able to add and remove elements.
    Also implement a method {\tt push(Integer incr)} that increments the root element by {\tt incr} and then pushes it (by swapping 
    values) to either the left or right branch.

    Set up a benchmark where you first add 64 elements with random values to a heap. The run a sequence of push operations 
    where you increment a by a random number ($\in[10,20]$). Collect some statistics on how deep the push operations need to go.
    
    Change the add method so that it also returns the depth. Run the same benchmark as above but now collect statistics on the 
    add method.

    \item Priority queue implementing a heap using an array. Benchmark the array implementation and see how it compares to the 
    \item linked list implementation.

\end{itemize}

\section{Method \& Theory}
\label{method}

\section{Discussion}


\newpage
\FloatBarrier
\section*{Code}
All the code can be found here: \href{https://github.com/adrian-jonsson-sjoedin/ID1021-AlgoData/tree/main/Tasks/Queues/src}{GitHub}

\begin{code}
    \captionof{listing}{Add and remove in first linked list queue}
    \label{code:queueNoLast}
    \begin{minted}{java}

\end{minted}
\end{code}

\begin{code}
    \captionof{listing}{Enqueue and dequeue in second linked list queue}
    \label{code:queue}
    \begin{minted}{java}

    \end{minted}
\end{code}

\begin{code}
    \captionof{listing}{The {\tt next()} method for the iterator}
    \label{code:iterator}
    \begin{minted}{java}

    \end{minted}
\end{code}

\begin{code}
    \captionof{listing}{Array implementation of {\tt enqueue(Item item)}}
    \label{code:arrayAdd}
    \begin{minted}{java}

    \end{minted}
\end{code}

\begin{code}
    \captionof{listing}{Array implementation of {\tt dequeue()}}
    \label{code:arrayDequeue}
    \begin{minted}{java}

    \end{minted}
\end{code}

\begin{code}
    \captionof{listing}{The {\tt resize} method}
    \label{code:resize}
    \begin{minted}{java}

    \end{minted}
\end{code}
\end{document}
