\documentclass[a4paper,11pt]{article}
\usepackage{graphicx}
\usepackage[utf8]{inputenc}
\usepackage{hyperref}
\usepackage{placeins}
\usepackage{minted}


\hypersetup{
    colorlinks=true,
    linkcolor=blue,
    filecolor=black,      
    urlcolor=blue,
    citecolor=black,
}
\begin{document}

\title{
    \textbf{Sorting Report.}
}
\author{Adrian Jonsson Sjödin}
\date{Fall 2022}

\maketitle

\section*{Introduction}
The task in this assignment was to explore some different sorting algorithms to gain a
better understanding how different implementations can affect the time it takes to sort
a large array.
\section*{Task}
Implement the three following sorting algorithms: Selection sort, Insert sort and Merge
sort. For each of the algorithms explain the run time as a function of the size of the
array and state their time complexity using Big O notation. Lastly do some benchmark and
compare how the different algorithms compares to each other.

\section*{Method}
The Selection Sort algorithm was the first to be implemented and is quite straight forward.
It takes the first element and then looks through the array to see if there's an element
smaller than it. When it has looked through the whole array it swaps the first element with
the found minimum element and then goes on to the next element and repeat the process
until the end of the array has been reached. This means that it will have a time complexity
of $\mathcal{O}(n^2)$ since it will have to take one element at a time (out of $n$ elements)
and look through an array of size $n$ to see if there's a smaller element $n-1$ times.





\section*{Result}



\section*{Discussion}



The code for this assignment can be found here \href{https://github.com/adrian-jonsson-sjoedin/ID1021-AlgoData/blob/main/Tasks/Sorting/src}{GitHub}
\end{document}
