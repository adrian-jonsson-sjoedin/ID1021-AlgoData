\documentclass[a4paper,11pt]{article}
\usepackage{graphicx}
\usepackage[utf8]{inputenc}
\usepackage{hyperref}
\usepackage{placeins}
\usepackage[newfloat]{minted}
\usepackage{caption}

\newenvironment{code}{\captionsetup{type=listing}}{}
\SetupFloatingEnvironment{listing}{name=Code Overview}


\hypersetup{
    colorlinks=true,
    linkcolor=blue,
    filecolor=black,      
    urlcolor=blue,
    citecolor=black,
}
\begin{document}

\title{
    \textbf{Trees Report.}
}
\author{Adrian Jonsson Sjödin}
\date{Fall 2022}

\maketitle

\section*{Introduction}
In this assignment we will take a closer look at \textit{tree} structures, and in
particular at tree structure called \textit{binary trees}. The operations that we will look at and
implement are: construction, adding and searching for and removing an item.

\section*{Task}
Implement a binary tree structure that is ordered with smaller keys to the left and create the
following two methods:
\begin{itemize}
    \item {\tt add(Integer key, Integer value)}: adds a new node (leaf) to the
          tree that maps the key to the value. If the key is already present we
          update the value of the node.
    \item {\tt lookup(Integer key)}: find and return the value associate to the
          key. If the key is not found we return null.
\end{itemize}
benchmark the lookup algorithm and compare it to the benchmark of the binary search algorithm that
was done in a previous assignment.

Also create an iterator that we can use to traverse the tree. The iterator should implements Java's
Iterator class and override the {\tt hasNext()} and {\tt next()} method. Explain how you implemented
the methods of the tree iterator class. Also describe what will happen if you create an iterator,
retrieve a few elements snd then add new elements to the tree. Will it work, what is the state of the
iterator, will we lose values?

\section*{Method \& Theory}
This data structure is called a \textit{tree} since the structure originates in a root from where we
spread out into branches, that then in turn can further divide into their own branches. A branch that
does not divide further is terminated by a so called leaf.

A \textit{binary tree} is a tree whose branch always divides into two branches, unless it terminates
into a leaf.

\begin{code}
    \captionof{listing}{Add integer to stack}
    \label{code:add}
    \begin{minted}{java}
 public void add(int value) {
    Node newHead = new Node(value, this.head);
    this.head = newHead;
    this.size++;
 }
\end{minted}
\end{code}


\section*{Result}

\section*{Discussion}

All the code can be found here: \href{https://github.com/adrian-jonsson-sjoedin/ID1021-AlgoData/tree/main/Tasks/LinkedLists/src}{GitHub}



\end{document}
