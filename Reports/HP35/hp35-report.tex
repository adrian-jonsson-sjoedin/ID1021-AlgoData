\documentclass[a4paper,11pt]{article}
\usepackage{graphicx}
\usepackage[utf8]{inputenc}
\usepackage{hyperref}

\usepackage{minted}


\hypersetup{
    colorlinks=true,
    linkcolor=blue,
    filecolor=black,      
    urlcolor=blue,
    citecolor=black,
}
\begin{document}

\title{
  \textbf{HP35 Calculator Report.}
}
\author{Adrian Jonsson Sjödin}
\date{Fall 2022}

\maketitle

\section*{Introduction}
In this assignment an old fashion calculator that uses reverse polish notation will be
implemented. The purpose of this is to gain an understanding of how a \textit{stack} is
constructed and works.

\section*{Task 1}

Complete the code for the different classes outlined by the teacher and implement the stack.
Implement both a static stack and a dynamic stack that can expand and shrink.
They questions that should be answered are:
\begin{itemize}
  \item Does the pointer point to the location above the top of the stack or does it
        point to the top of the stack?
  \item What is the value of the pointer when the stack is empty?
  \item What should you do when a program tries to push a value on a full stack?
  \item What should happen when someone pops an item from an empty stack?
\end{itemize}
Finally do some benchmarking of the different stack and examine how well the two implementations
works.
\section*{Method}

Completing the code for the different classes \textbf{Item, ItemType} and \textbf{Calculator}
was straightforward since it was just adding some constructor and getters, and complete the
case statement in the \textbf{Calculator} class.

The \textbf{Stack} class however needed to be implemented from scratch. This though is quite
straightforward for a static stack since all it need is to be able to push values to the stack,
pop values from the stack, and check wether or not the stack is empty or full. This is done
by using a pointer to index through the array that represent the stack as seen in the code bellow.


\begin{minted}{java}
public void push(int element) {
  if (isFull()) {
    throw new StackOverflowError("Stack is full");
  }
  arr[++pointer] = element;
}
\end{minted}
We check if the stack is full by simply looking at what value the pointer is set to, and as
long as it is not equal to the length of the array we are free to push to the stack. This is
done by first incrementing the pointer and then add the element to the stack. If the stack is
full we throw a StackOverflowError.

\begin{minted}{java}
public int pop() {
  if (isEmpty()) {
    throw new EmptyStackException();
}
  return arr[pointer--];
}
\end{minted}
Popping the stack is implemented in a similar way, with the difference being that we check
if the pointer is set to $-1$, which would mean that the stack is empty. If it is we throw a
EmptyStackException, otherwise we return the element that the pointer is currently pointing
to and \textit{then} decrement the pointer.

To implement a dynamic stack that can be rescaled one need only implement two new methods
in the stack class. Instead of throwing a StackOverflowError when {\tt isFull()} is called,
we instead call the method {\tt expandArray()}, which takes the current stack and copies it
to a new array with double the capacity.

Similarly to reduce it we add the method {\tt reduceSize()}, which looks at what the pointer
is pointing at and if it is less than half the stack capacity it reduces it by half the capacity.

\begin{minted}{java}
public void reduceSize() {
  int curr_length = index + 1;
  if (curr_length < capacity / 2) {
    int[] new_array = new int[capacity / 2];
    :        
    }
}
\end{minted}

\section*{Result}
To benchmark the different stacks we measured the best time out of $n$ runs that it took for the calculator to
calculate an expression with 500 values followed by 499 add operations.
\begin{table}[h]
  \begin{center}
    \begin{tabular}{|c|c|c|c|}
      \hline
      \textbf{Stack} & \textbf{Number of calculations} & \textbf{Time in ns} & \textbf{Time increase} \\
      \hline
      Static         & 1 000                           & 2439                & -                      \\
      Dynamic        & 1 000                           & 10010               & 4.1                    \\
      \hline
      Static         & 10 000                          & 2251                & -                      \\
      Dynamic        & 10 000                          & 6410                & 2.8                    \\
      \hline
      Static         & 1 000 000                       & 1702                & -                      \\
      Dynamic        & 1 000 000                       & 5033                & 2.9                    \\
      \hline
    \end{tabular}
    \caption{Stack benchmarks}
    \label{tab:task1}
  \end{center}
\end{table}


\section*{Discussion}

A static stack is as expected faster than a dynamic stack with a time complexity of
$\mathcal{O}(1)$, since it never needs to copy the whole stack over to a bigger/smaller
stack. So when it comes to just the time complexity of the different stacks the static
stack is better than the dynamic stack with its time complexity of $\mathcal{O}(n)$.
However the advantage of the dynamic stack is that it doesn't tie up as much memory as the
static stack, since it adjust itself depending on how much space is needed and can also be
implemented when one doesn't know exactly how much space might be required. However in cases
where one knows in advance how large a stack is required a static stack would be the better
alternative.
\section*{Task 2}

Use the calculator from task 1 to calculate the last digit in your social security number
according to the formula
\[ 10 - \left( (y_1 *'+y_2*' +m_1*'m_2*'...)mod_{10} \right) \]
The $*'$ operator means that any multiplication that gives $>10$ should instead be though of as
$a+b$. To be able to calculate your last digit implement this special operator as well as
the modulus operator into the calculator.

\section*{Method}

The $mod_{10}$ operation is simple to implement. We simply need to add it as a type in the
\textbf{ItemType} enum class and add the following to the case statement in the
\textbf{Calculator} class.

\begin{minted}{java}
case MOD -> {
  int y = stack.pop();
  stack.push(y % 10);
}
\end{minted}

The special operator is little bit tricker but not by much. What I implemented was a method
that takes the value at the top of the stack and add together the result of integer division
and modulo division by 10, and then pushes it back to the stack. This can be seen in the code
bellow.

\begin{minted}{java}
case MULTOADD -> {
  int x = stack.pop();
  int y = x / 10;
  int z = x % 10;
  stack.push(y + z);
} 
\end{minted}

Having implemented these two things we simply need to implement everything correctly
into the calculator.
\section*{Result}
Inputting,
\[
  10,x,x,*,*',x,x,x,x,x,*,*',x,x,x,*,*',+,+,+,+,+,+,mod
\]
where $x$ are my specific social security numbers left blanked for security reason, into the
calculator gives me my last digit.

\section*{Discussion}
When implementing the special operator I choose to implement a version that required that the
user knew in advanced that the previous multiplication operation would result in a value
greater than 10. I did this because of time constraint and thus choosing the easiest and
quickest operation. However for this particular task it could have been better to implement
it so that after every multiplication operation the program checked if that multiplication
resulted in a value greater than 10, and if it did then immediately do the special operation,
thus negating the need to enter the special operator into the calculator. This is however only
if the calculator's only job is to calculate the last digit of a social security number, since
it would break the multiplication operation for regular operations.

All code for this assignment can be found here: \href{https://github.com/adrian-jonsson-sjoedin/ID1021-AlgoData/tree/main/Tasks/HP35/HP35Calculator/src}{GitHub}


\end{document}
