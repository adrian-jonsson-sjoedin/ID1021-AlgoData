\documentclass[a4paper,11pt]{article}

\usepackage[utf8]{inputenc}

\usepackage{minted}

\begin{document}

\title{
  \textbf{My first report.}
}
\author{My Name}
\date{Spring Fall 2022}

\maketitle

\section*{Introduction}

This is what a report should look like, vanilla \LaTeX with regular
page width and height and single spaced lines.

The name of the report should not be "My first report" and the name
should not be "My Name". I thought that would be obvious but each year
I have submitted reports were these templates have not been changed.

\section*{Layout}

The rows in regular article mode are short - because it makes it
easier to read. Do not set the column width or margins explicitly, let
LaTeX decide what it should look like.

Don't use any fancy packages that will turn your report into a
Christmas tree, keep it simple!.

\subsection*{sections}

Since this is a small report I can omit having numbered sections and
you do this by using section commands that end with a {\tt *}. You can
of course have subsections etc.


\subsection*{inserting code}

Code snippets are included using the package {\tt minted}. If you
want to include a program statement in running text you can do this
using for example teletype-text: {\tt List.sort()}.

\begin{minted}{java}
  for (int i = 0; i < 100; i++){
    sum += i;
  }
\end{minted}

The reports that you hand in should be four pages long - but not four
pages of code! Use code snippets where you want to describe how things
are done but don't include code just because you have written it.

\section*{numbers}

You will include some run-time measurements in your reports. You
should the think about the number of significant figures that you
use. Just because a benchmark took $1.2345678 s$ does not mean that
you should report it in this way. If you write this in your report
you're implicitly saying - if I do this again the number will be the
same. This could be true but I doubt that anything you do on a
computer can be determined with an 8 figure accuracy. The next time
you try it might very well take $1.2354678 s$. What you report is
maybe $1.235 s$ or $1.2 s$?

\subsection*{tables}

Numbers are often best presented in a table. You will have to do some
reading on how to format tables but the general structures is quite
easy. This is for example a table with some run-time figures.

\begin{table}[h]
  \begin{center}
    \begin{tabular}{l|c|c}
      \textbf{prgm} & \textbf{runtime} & \textbf{ratio} \\
      \hline
      dummy         & 115              & 1.0            \\
      union         & 535              & 4.6            \\
      tailr         & 420              & 3.6            \\
    \end{tabular}
    \caption{Union and friends, list of 50000 elements, runtime in microseconds}
    \label{tab:table1}
  \end{center}
\end{table}

As you see in the table above, the run time per se might not be
interesting. The interesting thing is how it relates to something
else. Look at the ratios above, it gives you the information that we
are looking for. So when you include numbers, ask your self why you
have these numbers in the report. What is the purpose, can you
describe it in a better way?


\section*{no f*ing screen shots}

I know that you are all very happy that things actually work and
eagerly want to show what things look like on you screen but please,
don't use {\em screen shots}. It looks ugly and it's impossible to mark or
copy the things that you want to show. It also, most often, show a lot
of irrelevant things so instead of using an image, copy the text and
format it so it's easy to read.

\section*{graphs}

Once you start to generate graphs make sure that they are readable and
have sensible information on the axes.

There are many ways to generate graphs but you want to
use a way that minimize manual work. My tool over the years has been
  {\em Gnuplot} and if you do not have a favorite tool you could give it
a try.

Gnuplot is not a statistical program nor a program that is very good at
manipulating numbers but it is very good at taking a sequence of
numbers and generate a nice graph.

\section*{\LaTeX things}

Some \LaTeX errors that I frequently see that could easily be avoided
if you only know where they come from.

\section*{less than}

If you in your LaTeX code write "5 \textless\ 7" it will look like 5 <
7 and "9 \textgreater\ 7" will look like 9 > 7. Using the characters
\textless\ and \textgreater\ directly does not work ... so, how did I
do it?  I used the commands {\tt  \textbackslash textless} and {\tt
    \textbackslash textgreater} to generate the symbols \textless\ and
\textgreater.

You could also use {\tt \{\textbackslash tt 5 < 7\}} but then it
will use the teletype font and look like this: {\tt 5 < 7}.

Still another way is to write it using so called {\tt math mode}. This
is a mode used for writing mathematical formulas in a nice way. You
enclose your expression in {\tt \$} signs like this {\tt \$5 < 7\$}
and then it will look like this $5 < 7$.

If you have a larger mathematical expression you enclose it in double
\$ and the result is that it is written centered with some space
around it like this:  $$ 5 < (3 * 8 / 3 ) $$

\subsection*{why strange font}

If you want to write {\em foo} in teletype font you write like this
\verb+{\tt foo}+. If you forget the closing \} then it will look like
this: {\tt foo. Now everything here after until the end of you report
will look like this. }







\end{document}
