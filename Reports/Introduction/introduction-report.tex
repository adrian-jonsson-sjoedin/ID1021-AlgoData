\documentclass[a4paper,11pt]{article}

\usepackage[utf8]{inputenc}

\usepackage{minted}

\begin{document}

\title{
  \textbf{Introduction Task Report.}
}
\author{Adrian Jonsson Sjödin}
\date{Fall 2022}

\maketitle

\section*{Introduction}

The purpose of this task is to determine the time efficiency of three different operations
over an array of elements, as well as familiarizing oneself with writing a report in \LaTeX.

The three operations are:
\begin{itemize}
  \item Random access: reading or writing a value at a random location in an array.
  \item Search: searching through an array looking for an item.
  \item Duplicates: finding all common values in two arrays.
\end{itemize}

\section*{Task 1}

Set up a benchmark where the access method is called on a larger and larger array of size
\textit{n} and present your conclusions and observations on how much time is needed for a reading
of a value at a random location in the array.

\section*{Method}

To be able to measure the time it takes for one random access of an array of size
\textit{n}, I used the provided code from the task with some modifications. Mainly I adjusted
it so that the number of times we searched for an element in the array is not the same as
the size of the array. Instead we input the size of the array (\textit{n}) and the nr of
searches into the method call:

\begin{minted}{java}
  private static double access(int arraySize, int nrOfSearches) {
    int k = 1_000_000;
    int l = nrOfSearches;
   /* code here*/
  }
\end{minted}

This method will then return a time in nano seconds for the average time of one random array
access.

\section*{Result}

\begin{table}[h]
  \begin{center}
    \begin{tabular}{|c|c|c|}
      \hline
      \textbf{Array size} & \textbf{Nr. of searches} & \textbf{Time in ns} \\
      \hline
      10                  & 10 000                   & 0.86                \\
      100                 & 10 000                   & 0.26                \\
      1000                & 10 000                   & 0.26                \\
      10 000              & 10 000                   & 0.31                \\
      50 000              & 50 000                   & 0.50                \\
      100 000             & 10 000                   & 0.72                \\
      \hline
    \end{tabular}
    \caption{Output from the program being run once}
    \label{tab:task1}
  \end{center}
\end{table}


\section*{Discussion}

It seems like the time to access a random element in an array of
size \textit{n} increases with the size of the array, which I think is expected.
The time increase does not seem to be linear but rather a slower time increase than
that, which if true is good. However the trustworthiness of this result is
questionable since every time I ran the program it gave me quite different results,
where sometimes the time being lower and lower for larger \textit{n}, and other times
having one of the times in the middle being drastically larger than the rest.

I don't know if the reason behind this fluctuation of the time results are buggy code, or
if it is something that Java does behind the scenes that could somehow change the results.
It makes more sense that the reason lies in the code since one would think that whatever
Java might do behind the scenes should be consistent and thus not influence the results.
But even then I can't find where the problem would lie in the code which leads me to believe
that the reasons has to be connected to the randomness of the accessing somehow.

\section*{Task 2}

You will include some run-time measurements in your reports. You
should the think about the number of significant figures that you
use. Just because a benchmark took $1.2345678 s$ does not mean that
you should report it in this way. If you write this in your report
you're implicitly saying - if I do this again the number will be the
same. This could be true but I doubt that anything you do on a
computer can be determined with an 8 figure accuracy. The next time
you try it might very well take $1.2354678 s$. What you report is
maybe $1.235 s$ or $1.2 s$?

\subsection*{tables}

Numbers are often best presented in a table. You will have to do some
reading on how to format tables but the general structures is quite
easy. This is for example a table with some run-time figures.

\begin{table}[h]
  \begin{center}
    \begin{tabular}{l|c|c}
      \textbf{prgm} & \textbf{runtime} & \textbf{ratio} \\
      \hline
      dummy         & 115              & 1.0            \\
      union         & 535              & 4.6            \\
      tailr         & 420              & 3.6            \\
    \end{tabular}
    \caption{Union and friends, list of 50000 elements, runtime in microseconds}
    \label{tab:table1}
  \end{center}
\end{table}

As you see in the table above, the run time per se might not be
interesting. The interesting thing is how it relates to something
else. Look at the ratios above, it gives you the information that we
are looking for. So when you include numbers, ask your self why you
have these numbers in the report. What is the purpose, can you
describe it in a better way?


\section*{no f*ing screen shots}

I know that you are all very happy that things actually work and
eagerly want to show what things look like on you screen but please,
don't use {\em screen shots}. It looks ugly and it's impossible to mark or
copy the things that you want to show. It also, most often, show a lot
of irrelevant things so instead of using an image, copy the text and
format it so it's easy to read.

\section*{graphs}

Once you start to generate graphs make sure that they are readable and
have sensible information on the axes.

There are many ways to generate graphs but you want to
use a way that minimize manual work. My tool over the years has been
  {\em Gnuplot} and if you do not have a favorite tool you could give it
a try.

Gnuplot is not a statistical program nor a program that is very good at
manipulating numbers but it is very good at taking a sequence of
numbers and generate a nice graph.

\section*{\LaTeX things}

Some \LaTeX errors that I frequently see that could easily be avoided
if you only know where they come from.

\section*{less than}

If you in your LaTeX code write "5 \textless\ 7" it will look like 5 <
7 and "9 \textgreater\ 7" will look like 9 > 7. Using the characters
\textless\ and \textgreater\ directly does not work ... so, how did I
do it?  I used the commands {\tt  \textbackslash textless} and {\tt
    \textbackslash textgreater} to generate the symbols \textless\ and
\textgreater.

You could also use {\tt \{\textbackslash tt 5 < 7\}} but then it
will use the teletype font and look like this: {\tt 5 < 7}.

Still another way is to write it using so called {\tt math mode}. This
is a mode used for writing mathematical formulas in a nice way. You
enclose your expression in {\tt \$} signs like this {\tt \$5 < 7\$}
and then it will look like this $5 < 7$.

If you have a larger mathematical expression you enclose it in double
\$ and the result is that it is written centered with some space
around it like this:  $$ 5 < (3 * 8 / 3 ) $$

\subsection*{why strange font}

If you want to write {\em foo} in teletype font you write like this
\verb+{\tt foo}+. If you forget the closing \} then it will look like
this: {\tt foo. Now everything here after until the end of you report
will look like this. }







\end{document}
