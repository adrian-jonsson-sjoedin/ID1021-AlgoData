\documentclass[a4paper,11pt]{article}
\usepackage{graphicx}
\usepackage[utf8]{inputenc}
\usepackage{hyperref}
\usepackage{placeins}
\usepackage[newfloat]{minted}
\usepackage{caption}

\newenvironment{code}{\captionsetup{type=listing}}{}
\SetupFloatingEnvironment{listing}{name=Code Overview}


\hypersetup{
    colorlinks=true,
    linkcolor=blue,
    filecolor=black,      
    urlcolor=blue,
    citecolor=black,
}
\begin{document}

\title{
    \textbf{Hash Report.}
}
\author{Adrian Jonsson Sjödin}
\date{Fall 2022}

\maketitle

\section{Task}
\label{task}

\begin{itemize}
    \item 
    
    \item 

    \item 

\end{itemize}

\section{Method \& Theory}
\label{method}


\section{Result}

\begin{table}[h!]
    \begin{center}
        \begin{tabular}{|c|c|c|c|}
            \hline
            \textbf{push} & \textbf{Depth} & \textbf{add} & \textbf{Depth}\\
            \hline
             12        & 2    & 10    & 6                           \\
             17        & 3    & 19    & 6                           \\
             17        & 4    & 16    & 6                           \\
             12        & 3    & 10    & 3                           \\
             14        & 4    & 17    & 6                           \\
             20        & 3    & 10    & 3                           \\
             11        & 4    & 14    & 6                           \\
             15        & 3    & 11    & 6                           \\
             15        & 4    & 20    & 6                           \\
             13        & 4    & 19    & 6                           \\
             12        & 5    & 15    & 6                           \\
             13        & 4    & 11    & 6                           \\
             20        & 5    & 15    & 6                           \\
             13        & 4    & 17    & 6                           \\
             17        & 2    & 10    & 6                           \\
             17        & 5    & 16    & 6                           \\
             14         &4    & 11    & 6                           \\
             17         &5    & 19    & 5                           \\
             17         &5    & 11    & 6                           \\
             17         &5    & 11    & 4                           \\
            \hline
        \end{tabular}
        \caption{Benchmark for how deep the push and add method needed to go in the tree}
        \label{tab:tree}
    \end{center}
\end{table}
\FloatBarrier
\section{Discussion}




\newpage
\FloatBarrier
\section*{Code}
All the code can be found here: \href{https://github.com/adrian-jonsson-sjoedin/ID1021-AlgoData/tree/main/Tasks/Priority-Queues/src}{GitHub}

\begin{code}
    \captionof{listing}{$\mathcal{O}(n)$ {\tt add(Integer item)} method }
    \label{code:ListQueueAdd}
    \begin{minted}{java}

\end{minted}
\end{code}

\begin{code}
    \captionof{listing}{$\mathcal{O}(n)$ {\tt remove()} method}
    \label{code:ListQueueRemove}
    \begin{minted}{java}

    \end{minted}
\end{code}

\begin{code}
    \captionof{listing}{$\mathcal{O}(1)$ {\tt remove()} method}
    \label{code:QueueListRemove}
    \begin{minted}{java}

    \end{minted}
\end{code}

\begin{code}
    \captionof{listing}{$\mathcal{O}(n)$ {\tt add(Integer item)} method}
    \label{code:QueueListAdd}
    \begin{minted}{java}

    \end{minted}
\end{code}

\begin{code}
    \captionof{listing}{$\mathcal{O}(log(n))$ {\tt add(int priority, T item)} method}
    \label{code:TreeAdd}
    \begin{minted}{java}

    \end{minted}
\end{code}
\newpage
\begin{code}
    \captionof{listing}{$\mathcal{O}(log(n))$ {\tt remove()} method}
    \label{code:TreeRemove}
    \begin{minted}{java}

    \end{minted}
\end{code}

\begin{code}
    \captionof{listing}{The {\tt push(int increment)} method}
    \label{code:TreePush}
    \begin{minted}{java}

    \end{minted}
\end{code}

\begin{code}
    \captionof{listing}{Array Heap {\tt add(int item)} method}
    \label{code:ArrayAdd}
    \begin{minted}{java}

    \end{minted}
\end{code}
\begin{code}
    \captionof{listing}{Array Heap {\tt remove()} method}
    \label{code:ArrayRemove}
    \begin{minted}{java}

    \end{minted}
\end{code}

\end{document}
