\documentclass[a4paper,11pt]{article}
\usepackage{graphicx}
\usepackage[utf8]{inputenc}
\usepackage{hyperref}
\usepackage{placeins}
\usepackage[newfloat]{minted}
\usepackage{caption}

\newenvironment{code}{\captionsetup{type=listing}}{}
\SetupFloatingEnvironment{listing}{name=Code Overview}


\hypersetup{
    colorlinks=true,
    linkcolor=blue,
    filecolor=black,      
    urlcolor=blue,
    citecolor=black,
}
\begin{document}

\title{
    \textbf{Graph Report.}
}
\author{Adrian Jonsson Sjödin}
\date{Fall 2022}

\maketitle

\section{Task}
\label{task}
In this task we will read a file in CSV format that contains cities, there neighboring city and the time it takes to travel
between them. We will then implement different graph methods to find the shortest path between cities.
\begin{itemize}
    \item Take the CSV file and turn it into a graph (map) that we will later use to find the shortest path between cities. For this you will
          need two other classes {\tt City} and {\tt Connection}. Also create a quick lookup method that will be used to add cities
          to the map and when traversing the graph.

    \item Implement a simple program that finds the shortest path between two cities, regardless if loops and double back paths
          are present. Do some benchmarks and present the minimum path found, and how long it took to find the path. What are the
          limitations of this implementation?

    \item Implement another program that finds the shortest path but that can avoid loops by keeping track of which cities we have
          already passed. Rerun the benchmarks from the previous program and see if there's any improvements.

    \item Finally implement an improvement in which we use the found path to set a time limit for the future. If you have that to update
          the max value when you try different direct connection then you no longer ned to set a max value but can rather let it be null until
          you find a path. Do the benchmarks again and see if there is an improvement.

\end{itemize}

\section{Method \& Theory}
\label{method}
\subsection{Map}
We first created the different classes that was needed to create the map over the train system. These where the {\tt City} class,
the {\tt Connection} class and the {\tt Map} class itself.

The {\tt Connection} class is the simplest out of them and is used by the {\tt City} class to track which cities are
connected to that city. It consists of two private final fields {\tt City connectingCity} and {\tt Integer distance}, as well as a constructor
and two getters to retrieve the values stored in the fields.

The {\tt City} class is what will be used in the {\tt Map} class to build up the train network. It has two private fields {\tt String name}
and \\{\tt LinkedList<Connection> neighbors}, as well as a constructor to initialize the fields. It also have two getters to retrieve the
values of the private fields, and a public method called {\tt connect(City next, int distance)} used to add the neighboring stations for
city to the private field {\tt neighbors}. Using a linked list for that field allows the station to have as many connecting cities as needed,
and we don't have to know in advance the maximum amount of connecting stations. Furthermore adding them is just one line of code since we
can simply use the method {\tt add()} provided by the {\tt LinkedList} class.

Finally for the {\tt Map} class we have two private fields, {\tt City[] cities} to hold the train stations in, and {\tt final int mod = 541}
used for hashing the cities into the {\tt cities} array. When reading the CSV file we separate the line by the "," and store them in a string
array. Then we send the first city to our {\tt lookupOrAdd(String name)} method, then the second city and lastly use the {\tt connect}
method from {\tt City} class to connect city one to two and city two to one.

The {\tt lookupOrAdd} method takes the city's name and hash it and then first check if there is a city already stored at that index. If there's
not it add that city as a new city there. If there is something stored there we check if it is the same city as the city name parameter,
if it is then we want to return that city, if it is not then that means we have a collision and we then modify the index and redo the
process.

\subsection{Naive}
We have a functioning map over the train system, so next is to implement a search method that gives the shortest path between the cities.
This is done by using a \textit{breath first approach}. To avoid getting stuck in infinite loops we also have a parameter called {\tt max}
that is the maximum time we allow the trip to take. So when we search for a path between for example Malmö and Göteborg with a maximum allowed
time of 500, so will we start with going to the neighboring city that's first in the list and pass along (max - distance). Then go from
there to the next city that's first in that list, and so on all the way until we either find Malmö or the value (max - distance) that we
pass along becomes less than zero. If that happens we back up one city and try from there and so on. This is achieved in a similar way as
the depth first search in the binary tree assignment, in that we utilize recursive method calling. Doing it this way though means that we
can end up passing the same cities multiple times in some of the searches which will add to the execution time of the program.

\subsection{Paths}
The above search method works but it isn't very effective. In this search method we make sure to also track the cities that we have already
been to. This should make it so that we don't get stuck in any loops since we can never take a path that go back to a city that we have
already passed. This is implemented by simply adding the code provided in the assignment before the code that calls the method recursively.
An improvement that we can add is to also set the max time to the current found short time. This will eliminate all path that we haven't
yet traversed but that when we go to the first next station in will be longer than what we set short to.

\section{Result}

\begin{table}[h!]
    \begin{center}
        \caption{Search time for the {\tt Naive} class using a LinkedList and an array implementation of
                {\tt neighbors} in the {\tt City} class}
        \label{tab:Naive}
        \begin{tabular}{|c|c|c|c|}
            \hline
            \textbf{Route}         & \textbf{Travel time} & \textbf{LinkedList impl.} & \textbf{Array impl.} \\
            \hline
            Malmö to Göteborg      & 153 min              & 2 s                       & 1 ms                 \\
            Göteborg to Stockholm  & 211 min              & 3 ms                      & 2 ms                 \\
            Malmö to Stockholm     & 273 min              & 3 ms                      & 1 ms                 \\
            Stockholm to Sundsvall & 327 min              & 50 ms                     & 31 ms                \\
            Stockholm to Umeå      & 517 min              & 41,045 ms                 & 32,874 ms            \\
            Göteborg to Sundsvall  & 515 min              & 15,222 ms                 & 12,446 ms            \\
            Sundsvall to Umeå      & 190 min              & 1 ms                      & 1 ms                 \\
            Umeå to Göteborg       & 705 min              & 3 ms                      & 2                    \\
            Göteborg to Umeå       & Probably 705 min     & 10+ min                   & 10+ min              \\
            \hline
        \end{tabular}
    \end{center}
\end{table}

\begin{table}[h!]
    \begin{center}
        \caption{Search time for the {\tt Paths} class with and without the improvement regarding the
                {\tt max} variable}
        \label{tab:Paths}
        \begin{tabular}{|c|c|c|c|}
            \hline
            \textbf{Route}         & \textbf{Travel time} & \textbf{Before improv.} & \textbf{After improv.} \\
            \hline
            Malmö to Göteborg      & 153 min              & 303 ms                  & 2.6 ms                 \\
            Göteborg to Stockholm  & 211 min              & 156 ms                  & 1.3 ms                 \\
            Malmö to Stockholm     & 273 min              & 255  ms                 & 0.97 ms                \\
            Stockholm to Sundsvall & 327 min              & 183  ms                 & 13 ms                  \\
            Stockholm to Umeå      & 517 min              & 267  ms                 & 32 ms                  \\
            Göteborg to Sundsvall  & 515 min              & 252  ms                 & 14 ms                  \\
            Sundsvall to Umeå      & 190 min              & 533  ms                 & 378 ms                 \\
            Umeå to Göteborg       & 705 min              & 244  ms                 & 3.8 ms                 \\
            Göteborg to Umeå       & 705 min              & 280  ms                 & 59 ms                  \\
            Malmö to Kiruna        & 1,162 min            & 840  ms                 & 158 ms                 \\
            \hline
        \end{tabular}
    \end{center}
\end{table}

\FloatBarrier
\section{Discussion}


All the code can be found here: \href{https://github.com/adrian-jonsson-sjoedin/ID1021-AlgoData/tree/main/Tasks/Hash/src}{GitHub}

\end{document}
