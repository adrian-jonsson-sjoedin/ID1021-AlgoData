\documentclass[a4paper,11pt]{article}
\usepackage{graphicx}
\usepackage[utf8]{inputenc}
\usepackage{hyperref}
\usepackage{placeins}
\usepackage[newfloat]{minted}
\usepackage{caption}

\newenvironment{code}{\captionsetup{type=listing}}{}
\SetupFloatingEnvironment{listing}{name=Code Overview}


\hypersetup{
    colorlinks=true,
    linkcolor=blue,
    filecolor=black,      
    urlcolor=blue,
    citecolor=black,
}
\begin{document}

\title{
    \textbf{Graph Report.}
}
\author{Adrian Jonsson Sjödin}
\date{Fall 2022}

\maketitle

\section{Task}
\label{task}
In this task we will read a file in CSV format that contains cities, there neighboring city and the time it takes to travel
between them. We will then implement different graph methods to find the shortest path between cities.
\begin{itemize}
    \item Take the CSV file and turn it into a graph (map) that we will later use to find the shortest path between cities. For this you will
          need two other classes {\tt City} and {\tt Connection}. Also create a quick lookup method that will be used to add cities
          to the map and when traversing the graph.

    \item Implement a simple program that finds the shortest path between two cities, regardless if loops and double back paths
          are present. Do some benchmarks and present the minimum path found, and how long it took to find the path. What are the
          limitations of this implementation?

    \item Implement another program that finds the shortest path but that can avoid loops by keeping track of which cities we have
          already passed. Rerun the benchmarks from the previous program and see if there's any improvements.

    \item

\end{itemize}

\section{Method \& Theory}
\label{method}


\section{Result}

\begin{table}[h!]
    \begin{center}
        \caption{Linear search and binary search for when the zip code is in {\tt String} format}
        \label{tab:StringZip}
        \begin{tabular}{|c|c|c|}
            \hline
            linear 111 15 & Stockholm & 92 ns    \\
            linear 984 99 & Pajala    & 32224 ns \\
            binary 111 15 & Stockholm & 301 ns   \\
            binary 984 99 & Pajala    & 141 ns   \\
            \hline
        \end{tabular}
    \end{center}
\end{table}

\begin{table}[h!]
    \begin{center}
        \caption{Linear search and binary search for when the zip code is in {\tt Integer} format}
        \label{tab:IntegerZip}
        \begin{tabular}{|c|c|c|}
            \hline
            linear 111 15 & Stockholm & 53 ns    \\
            linear 984 99 & Pajala    & 11318 ns \\
            binary 111 15 & Stockholm & 126 ns   \\
            binary 984 99 & Pajala    & 136 ns   \\
            \hline
        \end{tabular}
    \end{center}
\end{table}
\begin{table}[h!]
    \begin{center}
        \caption{Lookup for when the zip-codes are used as indexes}
        \label{tab:ZipIndexes}
        \begin{tabular}{|c|c|c|}
            \hline
            lookup 111 15 & Stockholm & 51 ns \\
            lookup 984 99 & Pajala    & 53 ns \\
            \hline
        \end{tabular}
    \end{center}
\end{table}
\begin{table}[h!]
    \begin{center}
        \caption{Lookup for when using Hash Buckets}
        \label{tab:BucketTime}
        \begin{tabular}{|c|c|c|}
            \hline
            lookup 111 15 & Stockholm & 158 ns \\
            lookup 984 99 & Pajala    & 69 ns  \\
            \hline
        \end{tabular}
    \end{center}
\end{table}
\begin{table}[h!]
    \begin{center}
        \caption{Number of collisions when using Hash Buckets}
        \label{tab:BucketCollisions}
        \begin{tabular}{|c|c|c|c|c|c|}
            \hline
            Modulo & Unique & 2 on same & 3 on same & 4 on same & Utilization \\
            \hline
            31327  & 8961   & 688       & 25        & 0         & 29\%        \\
            28627  & 8878   & 770       & 26        & 0         & 31\%        \\
            27773  & 8820   & 841       & 13        & 0         & 32\%        \\
            \hline
        \end{tabular}
    \end{center}
\end{table}
\begin{table}[h!]
    \begin{center}
        \caption{Lookup for "Slightly better"}
        \label{tab:HashTime}
        \begin{tabular}{|c|c|c|}
            \hline
            lookup 111 15 & Stockholm & 207 ns \\
            lookup 984 99 & Pajala    & 63 ns  \\
            \hline
        \end{tabular}
    \end{center}
\end{table}
\begin{table}[h!]
    \begin{center}
        \caption{Steps in "Slightly better"}
        \label{tab:Steps}
        \begin{tabular}{|cc|}
            \hline
            Max Steps                              & 59  \\
            Average steps when stepping            & 8.9 \\
            Average numb. of times we need to step & 1.1 \\
            \hline
        \end{tabular}
    \end{center}
\end{table}

\FloatBarrier
\section{Discussion}


All the code can be found here: \href{https://github.com/adrian-jonsson-sjoedin/ID1021-AlgoData/tree/main/Tasks/Hash/src}{GitHub}

\end{document}
